\chapter{Regulator PID}
\label{zad5}



\section{Algorytm działania}
Algorytm działania regulatora oraz implementacja została dobrze udokumentowana w pliku \verb+doPID.m +. Listing jego częsci algorytmicznej przedstawiony jest poniżej:
\begin{lstlisting}[style=custommatlab,frame=single,label={zad4_sim_lst},caption={Implementacja regulatora PID},captionpos=b]

function [ error ] = doPID( paras )   % Tylko dla auto

% Ustawiamy dlugosc symulacji
sim_len=1200;

% Tylko dla auto
K=paras(1);
Ti=paras(2);
Td=paras(3);

% Czas probkowania
T=1;

% Parametry wygodnego, dyskretnego PIDa
r0=K*(1+T/(2*Ti)+Td/T);
r1=K*(T/(2*Ti)-(2*Td/T)-1);
r2=K*Td/T;

% Inicjalizujemy macierze przechowujace zmienne
Y=zeros(sim_len,1);
U=zeros(sim_len,1);
e=zeros(sim_len,1);
y=zeros(sim_len,1);
u=zeros(sim_len,1);
Yzad=zeros(sim_len,1);
kk=linspace(1,sim_len,sim_len)';

% Ustalamy wartosci przed rozpoczeciem symulacji na wartosci w punktu pracy
Ypp=0.8;
Upp=2.0;
Y(1:11)=Ypp;
U(1:11)=Upp;

% Tworzymy horyzont wartosci zadanej
Yzad(1:29)=0.8;
Yzad(30:sim_len/3-1)=1.0;
Yzad(sim_len/3:2*sim_len/3-1)=0.6;
Yzad(2*sim_len/3:sim_len)=0.7;

%Rzutujemy ograniczenia sterowan wzgledem punktu pracy.
Umin=1.2;
Umax=2.8;
deltaumax=0.25;
umin=Umin-Upp;
umax=Umax-Upp;


for k=12:sim_len
    % Symulujemy wyjscie obiektu    
    Y(k)=symulacja_obiektu4Y(U(k-10),U(k-11),Y(k-1),Y(k-2));
    % Rzutujemy wartosc wyjscia wzgledem punktu pracy    
    y(k)=Y(k)-Ypp;
    % Liczymy uchyb i uaktualniamy wspolczynnik bledu
    e(k)=Yzad(k)-Y(k);
    error=error+e(k)^2;
    % Liczymy wartosc sterowania    
    u_wyliczone=r2*e(k-2)+r1*e(k-1)+r0*e(k)+u(k-1);

    % Rzutowanie ograniczen na wartosc sterowania
    if u_wyliczone<umin
        u_wyliczone=umin;
    elseif u_wyliczone>umax
        u_wyliczone=umax;
    end
    % Rzutowanie ograniczen na wartosc zmiany sterowania
    if u_wyliczone-u(k-1)<-deltaumax
        u_wyliczone=u(k-1)-deltaumax;
    elseif u_wyliczone-u(k-1)>deltaumax
        u_wyliczone=u(k-1)+deltaumax;
    end
   % Rzutowanie sterowania wzgledem punktu pracy
    u(k)=u_wyliczone;
    U(k)=u_wyliczone+Upp;
    
 end


\end{lstlisting}

\section{Ręczne strojenie regulatora PID}

W celu dobrania wstępnych parametrów regulatora PID użyto metody eksperymentalnej; przeprowadzono dużą ilosc symulacji dla arbitralnych wartosci $K_{\mathrm{r}}$,  $T_{\mathrm{i}}$, $T_{\mathrm{d}}$. Sposrod ponad 100 symulacji, wybrano te parametry, których wskaźnik jakoci był najlepszy. Stało się to dla regulatora przedstawionego na rysunku \ref{PID_0}.
\begin{figure}[t]
    \centering
    \begin{tikzpicture}
    \begin{axis}[
    width=0.98\textwidth,
    xmin=0,xmax=1200,ymin=0.4, ymax=1.2,
    xlabel={$k$},
    ylabel={$y[k]$},
    xtick={0, 300,600,900,1200},
    ytick={0.4,0.6,0.8,1.0,1.2},
    legend pos=south east,
    y tick label style={/pgf/number format/1000 sep=},
    ]
    \addplot[blue, semithick] file{rysunki/PROJ1_PID_EXPerror=6.81K=1Ti=8Td=1.txt};
    \addplot[red, semithick] file{rysunki/PROJ1_4_Yzad.txt};
    \addlegendentry{$y^{zad}[k]$},
    \addlegendentry{$y[k]$},
    \addlegendimage{no markers, blue}
	\addlegendimage{no markers, red}
    \end{axis}
    \end{tikzpicture}
    \caption{Niegasnące oscylacje wyjścia obiektu przy wzmocnieniu krytycznym}
    \label{zad5_niegasnace_oscylacje}
\end{figure}

\begin{figure}[b]
    \centering
    \begin{tikzpicture}
    \begin{axis}[
    width=0.98\textwidth,
    xmin=0,xmax=1200,ymin=1.2, ymax=2.8,
    xlabel={$k$},
    ylabel={$u[k]$},
    xtick={0, 300,600,900,1200},
    ytick={1.2,1.6,2.0,2.4,2.8},
    legend pos=south east,
    y tick label style={/pgf/number format/1000 sep=},
    ]
    \addplot[const plot, blue, semithick] file{rysunki/PROJ1_PID_EXP_STERerror=6.81K=1Ti=8Td=1.txt};
    \legend{$u[k]$}
    \end{axis}
    \end{tikzpicture}
    \caption{Przebieg sygnału sterującego }
    \label{zad5_niegasnace_oscylacje_ster}
\end{figure}
\FloatBarrier


