\chapter{Regulator PID}
\label{zad5}



\section{Algorytm działania}
Algorytm działania regulatora oraz implementacja została dobrze udokumentowana w pliku \verb+doPID.m +. Listing jego częsci algorytmicznej przedstawiony jest poniżej:
\begin{lstlisting}[style=custommatlab,frame=single,label={zad4_sim_lst},caption={Implementacja regulatora PID},captionpos=b]

function [ error ] = doPID( paras )   % Tylko dla auto

% Ustawiamy dlugosc symulacji
sim_len=1200;

% Tylko dla auto
K=paras(1);
Ti=paras(2);
Td=paras(3);

% Czas probkowania
T=1;

% Parametry wygodnego, dyskretnego PIDa
r0=K*(1+T/(2*Ti)+Td/T);
r1=K*(T/(2*Ti)-(2*Td/T)-1);
r2=K*Td/T;

% Inicjalizujemy macierze przechowujace zmienne
Y=zeros(sim_len,1);
U=zeros(sim_len,1);
e=zeros(sim_len,1);
y=zeros(sim_len,1);
u=zeros(sim_len,1);
Yzad=zeros(sim_len,1);
kk=linspace(1,sim_len,sim_len)';

% Ustalamy wartosci przed rozpoczeciem symulacji na wartosci w punktu pracy
Ypp=0.8;
Upp=2.0;
Y(1:11)=Ypp;
U(1:11)=Upp;

% Tworzymy horyzont wartosci zadanej
Yzad(1:29)=0.8;
Yzad(30:sim_len/3-1)=1.0;
Yzad(sim_len/3:2*sim_len/3-1)=0.6;
Yzad(2*sim_len/3:sim_len)=0.7;

%Rzutujemy ograniczenia sterowan wzgledem punktu pracy.
Umin=1.2;
Umax=2.8;
deltaumax=0.25;
umin=Umin-Upp;
umax=Umax-Upp;


for k=12:sim_len
    % Symulujemy wyjscie obiektu    
    Y(k)=symulacja_obiektu4Y(U(k-10),U(k-11),Y(k-1),Y(k-2));
    % Rzutujemy wartosc wyjscia wzgledem punktu pracy    
    y(k)=Y(k)-Ypp;
    % Liczymy uchyb i uaktualniamy wspolczynnik bledu
    e(k)=Yzad(k)-Y(k);
    error=error+e(k)^2;
    % Liczymy wartosc sterowania    
    u_wyliczone=r2*e(k-2)+r1*e(k-1)+r0*e(k)+u(k-1);

    % Rzutowanie ograniczen na wartosc sterowania
    if u_wyliczone<umin
        u_wyliczone=umin;
    elseif u_wyliczone>umax
        u_wyliczone=umax;
    end
    % Rzutowanie ograniczen na wartosc zmiany sterowania
    if u_wyliczone-u(k-1)<-deltaumax
        u_wyliczone=u(k-1)-deltaumax;
    elseif u_wyliczone-u(k-1)>deltaumax
        u_wyliczone=u(k-1)+deltaumax;
    end
   % Rzutowanie sterowania wzgledem punktu pracy
    u(k)=u_wyliczone;
    U(k)=u_wyliczone+Upp;
    
 end


\end{lstlisting}

\section{Ręczne strojenie regulatora PID}

\subsection{Wyliczenie wstępnych parametrów regulatora}
\label{PID_START}
Celem naszej regulacji jest minimalizacja wskaźnika jakosci, będącym sumą kwadratów uchybu regulacji w każdym punkcie. W celu dobrania wstępnych parametrów regulatora PID użyto metody eksperymentalnej; przeprowadzono dużą ilosc symulacji dla arbitralnych wartosci $K_{\mathrm{r}}$,  $T_{\mathrm{i}}$, $T_{\mathrm{d}}$. Sposrod ponad 100 symulacji, wybrano te parametry, których wskaźnik jakoci był najlepszy. Stało się to dla regulatora przedstawionego na rysunku \ref{PID_0}.
\begin{figure}[t]
    
    \textbf{E=\num{6.81}}
    \centering
    \begin{tikzpicture}
    \begin{axis}[
    width=0.98\textwidth,
    xmin=0,xmax=1200,ymin=0.4, ymax=1.2,
    xlabel={$k$},
    ylabel={$y[k]$},
    xtick={0, 300,600,900,1200},
    ytick={0.4,0.6,0.8,1.0,1.2},
    legend pos=south east,
    y tick label style={/pgf/number format/1000 sep=},
    ]
    \addplot[blue, semithick] file{rysunki/PROJ1_PID_EXPerror=6.81K=1Ti=4Td=0.5.txt};
    \addplot[red, semithick] file{rysunki/PROJ1_4_Yzad.txt};
    \addlegendentry{$y[k]$},
    \addlegendentry{$y^{zad}[k]$},
    \addlegendimage{no markers, blue}
	\addlegendimage{no markers, red}
    \end{axis}
    \end{tikzpicture}
    \caption{Wyjscie dla $K_{\mathrm{r}}$=\num{1},  $T_{\mathrm{i}}$=\num{4}, $T_{\mathrm{d}}$=\num{0.5}}
    \label {PID_0}
\end{figure}

\begin{figure}[b]
    \centering
    \begin{tikzpicture}
    \begin{axis}[
    width=0.98\textwidth,
    xmin=0,xmax=1200,ymin=1.2, ymax=2.8,
    xlabel={$k$},
    ylabel={$u[k]$},
    xtick={0, 300,600,900,1200},
    ytick={1.2,1.6,2.0,2.4,2.8},
    legend pos=south east,
    y tick label style={/pgf/number format/1000 sep=},
    ]
    \addplot[const plot, blue, semithick] file{rysunki/PROJ1_PID_EXP_STERerror=6.81K=1Ti=4Td=0.5.txt};
    \legend{$u[k]$}
    \end{axis}
    \end{tikzpicture}
    \caption{Sterowanie dla $K_{\mathrm{r}}$=\num{1},  $T_{\mathrm{i}}$=\num{4}, $T_{\mathrm{d}}$=\num{0.5} }
    
\end{figure}
\FloatBarrier


\subsection{Zmniejszenie przeregulowania}

W celu zmniejszenia znacznego przeregulowania występującego w regulatorze zmniejszono jego wzmocnienie do  $K_{\mathrm{r}}$=\num{0.8} oraz dodatkowo zmieniono parametr członu różniczkującego do $T_{\mathrm{d}}$=\num{0.25}. Wyniki symulacji dla zmienionego regulatora przedstawone zostały na rysunku \ref{PID_1}.


\begin{figure}[t]
    
    \textbf{E=\num{6.51}}
    \centering
    \begin{tikzpicture}
    \begin{axis}[
    width=0.98\textwidth,
    xmin=0,xmax=1200,ymin=0.4, ymax=1.2,
    xlabel={$k$},
    ylabel={$y[k]$},
    xtick={0, 300,600,900,1200},
    ytick={0.4,0.6,0.8,1.0,1.2},
    legend pos=south east,
    y tick label style={/pgf/number format/1000 sep=},
    ]
    \addplot[blue, semithick] file{rysunki/PROJ1_PID_EXPerror=6.51K=0.8Ti=4Td=0.25.txt};
    \addplot[red, semithick] file{rysunki/PROJ1_4_Yzad.txt};
    \addlegendentry{$y[k]$},
    \addlegendentry{$y^{zad}[k]$},
    \addlegendimage{no markers, blue}
	\addlegendimage{no markers, red}
    \end{axis}
    \end{tikzpicture}
    \caption{Wyjscie dla $K_{\mathrm{r}}$=\num{0.8},  $T_{\mathrm{i}}$=\num{4}, $T_{\mathrm{d}}$=\num{0.25}}
    \label {PID_1}
\end{figure}

\begin{figure}[b]
    \centering
    \begin{tikzpicture}
    \begin{axis}[
    width=0.98\textwidth,
    xmin=0,xmax=1200,ymin=1.2, ymax=2.8,
    xlabel={$k$},
    ylabel={$u[k]$},
    xtick={0, 300,600,900,1200},
    ytick={1.2,1.6,2.0,2.4,2.8},
    legend pos=south east,
    y tick label style={/pgf/number format/1000 sep=},
    ]
    \addplot[const plot, blue, semithick] file{rysunki/PROJ1_PID_EXP_STERerror=6.51K=0.8Ti=4Td=0.25.txt};
    \legend{$u[k]$}
    \end{axis}
    \end{tikzpicture}
    \caption{Sterowanie dla $K_{\mathrm{r}}$=\num{0.8},  $T_{\mathrm{i}}$=\num{4}, $T_{\mathrm{d}}$=\num{0.25} }
    
\end{figure}
\FloatBarrier

\subsection{Dalsze zmniejszenie przeregulowania}

Przeregulowanie ciągle występuję w dosc znacznym stopniu, dlatego w następnym kroku zmneijeszono całkowanie do $T_{\mathrm{i}}$=\num{5} oraz dodatkowo zmniejszono różniczkowanie do 
$T_{\mathrm{d}}$=\num{0.05}. Wyniki symulacji dla zmienionego regulatora przedstawione zostały na rysunku \ref{PID_2}.


\begin{figure}[t]
    
    \textbf{E=\num{5.95}}
    \centering
    \begin{tikzpicture}
    \begin{axis}[
    width=0.98\textwidth,
    xmin=0,xmax=1200,ymin=0.4, ymax=1.2,
    xlabel={$k$},
    ylabel={$y[k]$},
    xtick={0, 300,600,900,1200},
    ytick={0.4,0.6,0.8,1.0,1.2},
    legend pos=south east,
    y tick label style={/pgf/number format/1000 sep=},
    ]
    \addplot[blue, semithick] file{rysunki/PROJ1_PID_EXPerror=5.95K=0.8Ti=5Td=0.05.txt};
    \addplot[red, semithick] file{rysunki/PROJ1_4_Yzad.txt};
    \addlegendentry{$y[k]$},
    \addlegendentry{$y^{zad}[k]$},
    \addlegendimage{no markers, blue}
	\addlegendimage{no markers, red}
    \end{axis}
    \end{tikzpicture}
    \caption{Wyjscie dla $K_{\mathrm{r}}$=\num{0.8},  $T_{\mathrm{i}}$=\num{5}, $T_{\mathrm{d}}$=\num{0.05}}
    \label {PID_2}
\end{figure}

\begin{figure}[b]
    \centering
    \begin{tikzpicture}
    \begin{axis}[
    width=0.98\textwidth,
    xmin=0,xmax=1200,ymin=1.2, ymax=2.8,
    xlabel={$k$},
    ylabel={$u[k]$},
    xtick={0, 300,600,900,1200},
    ytick={1.2,1.6,2.0,2.4,2.8},
    legend pos=south east,
    y tick label style={/pgf/number format/1000 sep=},
    ]
    \addplot[const plot, blue, semithick] file{rysunki/PROJ1_PID_EXP_STERerror=5.95K=0.8Ti=5Td=0.05.txt};
    \legend{$u[k]$}
    \end{axis}
    \end{tikzpicture}
    \caption{Sterowanie dla $K_{\mathrm{r}}$=\num{0.8},  $T_{\mathrm{i}}$=\num{5}, $T_{\mathrm{d}}$=\num{0.95} }
    
\end{figure}
\FloatBarrier


\subsection{Przyspieszenie regulatora}

W celu zwiększenia szybkosci regulatora zwiększono jego wzmocnienie do  $K_{\mathrm{r}}$=\num{2.4} oraz w ramach kompensacji zmniejszono całkowanie do $T_{\mathrm{i}}$=\num{8}. Wyniki symulacji dla zmienionego regulatora przedstawione zostały na rysunku \ref{PID_3}.


\begin{figure}[t]
    
    \textbf{E=\num{5.14}}
    \centering
    \begin{tikzpicture}
    \begin{axis}[
    width=0.98\textwidth,
    xmin=0,xmax=1200,ymin=0.4, ymax=1.2,
    xlabel={$k$},
    ylabel={$y[k]$},
    xtick={0, 300,600,900,1200},
    ytick={0.4,0.6,0.8,1.0,1.2},
    legend pos=south east,
    y tick label style={/pgf/number format/1000 sep=},
    ]
    \addplot[blue, semithick] file{rysunki/PROJ1_PID_EXPerror=5.14K=2.4Ti=8Td=0.05.txt};
    \addplot[red, semithick] file{rysunki/PROJ1_4_Yzad.txt};
    \addlegendentry{$y[k]$},
    \addlegendentry{$y^{zad}[k]$},
    \addlegendimage{no markers, blue}
	\addlegendimage{no markers, red}
    \end{axis}
    \end{tikzpicture}
    \caption{Wyjscie dla $K_{\mathrm{r}}$=\num{0.8},  $T_{\mathrm{i}}$=\num{5}, $T_{\mathrm{d}}$=\num{0.05}}
    \label {PID_3}
\end{figure}

\begin{figure}[b]
    \centering
    \begin{tikzpicture}
    \begin{axis}[
    width=0.98\textwidth,
    xmin=0,xmax=1200,ymin=1.2, ymax=2.8,
    xlabel={$k$},
    ylabel={$u[k]$},
    xtick={0, 300,600,900,1200},
    ytick={1.2,1.6,2.0,2.4,2.8},
    legend pos=south east,
    y tick label style={/pgf/number format/1000 sep=},
    ]
    \addplot[const plot, blue, semithick] file{rysunki/PROJ1_PID_EXP_STERerror=5.14K=2.4Ti=8Td=0.05.txt};
    \legend{$u[k]$}
    \end{axis}
    \end{tikzpicture}
    \caption{Sterowanie dla $K_{\mathrm{r}}$=\num{0.8},  $T_{\mathrm{i}}$=\num{5}, $T_{\mathrm{d}}$=\num{0.05} }
    
\end{figure}
\FloatBarrier

\subsection{Człon różniczkujący}

Podczas wczesniejszego strojenia zauważono fakt, iż zmniejszenie członu różniczkującego pozytywnie wpływa na jakosć regulacji. Postanowiono więc znacznie go zmniejszyć, do wartosci $T_{\mathrm{d}}$=\num{0.0005}. Wyniki symulacji dla zmienionego regulatora przedstawione zostały na rysunku \ref{PID_4}.


\begin{figure}[t]
    
    \textbf{E=\num{4.94}}
    \centering
    \begin{tikzpicture}
    \begin{axis}[
    width=0.98\textwidth,
    xmin=0,xmax=1200,ymin=0.4, ymax=1.2,
    xlabel={$k$},
    ylabel={$y[k]$},
    xtick={0, 300,600,900,1200},
    ytick={0.4,0.6,0.8,1.0,1.2},
    legend pos=south east,
    y tick label style={/pgf/number format/1000 sep=},
    ]
    \addplot[blue, semithick] file{rysunki/PROJ1_PID_EXPerror=4.94K=2.4Ti=8Td=5e-05.txt};
    \addplot[red, semithick] file{rysunki/PROJ1_4_Yzad.txt};
    \addlegendentry{$y[k]$},
    \addlegendentry{$y^{zad}[k]$},
    \addlegendimage{no markers, blue}
	\addlegendimage{no markers, red}
    \end{axis}
    \end{tikzpicture}
    \caption{Wyjscie dla $K_{\mathrm{r}}$=\num{0.8},  $T_{\mathrm{i}}$=\num{5}, $T_{\mathrm{d}}$=\num{0.05}}
    \label {PID_4}
\end{figure}

\begin{figure}[b]
    \centering
    \begin{tikzpicture}
    \begin{axis}[
    width=0.98\textwidth,
    xmin=0,xmax=1200,ymin=1.2, ymax=2.8,
    xlabel={$k$},
    ylabel={$u[k]$},
    xtick={0, 300,600,900,1200},
    ytick={1.2,1.6,2.0,2.4,2.8},
    legend pos=south east,
    y tick label style={/pgf/number format/1000 sep=},
    ]
    \addplot[const plot, blue, semithick] file{rysunki/PROJ1_PID_EXP_STERerror=4.94K=2.4Ti=8Td=5e-05.txt};
    \legend{$u[k]$}
    \end{axis}
    \end{tikzpicture}
    \caption{Sterowanie dla $K_{\mathrm{r}}$=\num{0.8},  $T_{\mathrm{i}}$=\num{5}, $T_{\mathrm{d}}$=\num{0.05} }
    
\end{figure}
\FloatBarrier

\subsection{Końcowe dostrojenie}

Analogicznie jak w kroku \ref{PID_START} przeprowadzono dużą liczbę symulacji dla wielu kombinacji paramterów, nieznacznie oddalonych od obecnych. W ten sposób znaleziono ich optymalną wartosc równą: $K_{\mathrm{r}}$=\num{3.3}, $T_{\mathrm{i}}$=\num{9.25}, $T_{\mathrm{d}}$=\num{0.0005}. Wyniki symulacji dla takiego regulatora zostały przedstawione na rysunku \ref{PID_5}.

\begin{figure}[t]
    
    \textbf{E=\num{4.82}}
    \centering
    \begin{tikzpicture}
    \begin{axis}[
    width=0.98\textwidth,
    xmin=0,xmax=1200,ymin=0.4, ymax=1.2,
    xlabel={$k$},
    ylabel={$y[k]$},
    xtick={0, 300,600,900,1200},
    ytick={0.4,0.6,0.8,1.0,1.2},
    legend pos=south east,
    y tick label style={/pgf/number format/1000 sep=},
    ]
    \addplot[blue, semithick] file{rysunki/PROJ1_PID_EXPerror=4.82K=3.3Ti=9.25Td=0.0005.txt};
    \addplot[red, semithick] file{rysunki/PROJ1_4_Yzad.txt};
    \addlegendentry{$y[k]$},
    \addlegendentry{$y^{zad}[k]$},
    \addlegendimage{no markers, blue}
	\addlegendimage{no markers, red}
    \end{axis}
    \end{tikzpicture}
    \caption{Wyjscie dla $K_{\mathrm{r}}$=\num{0.8},  $T_{\mathrm{i}}$=\num{5}, $T_{\mathrm{d}}$=\num{0.05}}
    \label {PID_4}
\end{figure}

\begin{figure}[b]
    \centering
    \begin{tikzpicture}
    \begin{axis}[
    width=0.98\textwidth,
    xmin=0,xmax=1200,ymin=1.2, ymax=2.8,
    xlabel={$k$},
    ylabel={$u[k]$},
    xtick={0, 300,600,900,1200},
    ytick={1.2,1.6,2.0,2.4,2.8},
    legend pos=south east,
    y tick label style={/pgf/number format/1000 sep=},
    ]
    \addplot[const plot, blue, semithick] file{rysunki/PROJ1_PID_EXP_STERerror=4.82K=3.3Ti=9.25Td=0.0005.txt};
    \legend{$u[k]$}
    \end{axis}
    \end{tikzpicture}
    \caption{Sterowanie dla $K_{\mathrm{r}}$=\num{0.8},  $T_{\mathrm{i}}$=\num{5}, $T_{\mathrm{d}}$=\num{0.05} }
    
\end{figure}
\FloatBarrier


\section{Automatyczne strojenie regulatora PID}

Do automatycznego strojenia regulatora PID użyto już wczeniej używanej funkcji \verb+doPID.m+, której użyto jako funkcji liczącej wartosć wskaźnik jakosci (funkcja ją zwraca) w maltabowskim algorytmie \verb+fmincon+, której użyto w skrypcie \verb+PIDOptimizer.m+. Przeprowadzono dwie próby: dla małych wartosci parametrów początkowych oraz dla dużych wartosci kryteriów początkowych, co widać w listiingu \ref{LIST_1} poniżej.

\begin{lstlisting}[style=custommatlab,frame=single,label={zad4_sim_lst},caption={Implementacja automatycznego strojenia PID},captionpos=b]

x01=[1,1,1];
x02=[20,20,20];
lb=[0.01,0.01,0.01];
ub=[50,50,50];
paras=fmincon(@doPID,x0,[],[],[],[],lb,ub);

\label{LIST_1}
\end{lstlisting}

Wyniki pracy dla pierwszego regulatora przedstawiono na rysunku \ref{PID_AUTO1}, a drugiego na rysunku \ref{PID_AUTO2}.

\begin{figure}[t]
    
    \textbf{E=\num{4.84}}
    \centering
    \begin{tikzpicture}
    \begin{axis}[
    width=0.98\textwidth,
    xmin=0,xmax=1200,ymin=0.4, ymax=1.2,
    xlabel={$k$},
    ylabel={$y[k]$},
    xtick={0, 300,600,900,1200},
    ytick={0.4,0.6,0.8,1.0,1.2},
    legend pos=south east,
    y tick label style={/pgf/number format/1000 sep=},
    ]
    \addplot[blue, semithick] file{rysunki/PROJ1_PID_OPTK=3.2526Ti=18.6051Td=0.01error=4.8389.txt};
    \addplot[red, semithick] file{rysunki/PROJ1_4_Yzad.txt};
    \addlegendentry{$y[k]$},
    \addlegendentry{$y^{zad}[k]$},
    \addlegendimage{no markers, blue}
    \addlegendimage{no markers, red}
    \end{axis}
    \end{tikzpicture}
    \caption{Wyjscie dla $K_{\mathrm{r}}$=\num{3.25},  $T_{\mathrm{i}}$=\num{9.3}, $T_{\mathrm{d}}$=\num{0.005}}
    \label {PID_AUTO1}
\end{figure}

\begin{figure}[b]
    \centering
    \begin{tikzpicture}
    \begin{axis}[
    width=0.98\textwidth,
    xmin=0,xmax=1200,ymin=1.2, ymax=2.8,
    xlabel={$k$},
    ylabel={$u[k]$},
    xtick={0, 300,600,900,1200},
    ytick={1.2,1.6,2.0,2.4,2.8},
    legend pos=south east,
    y tick label style={/pgf/number format/1000 sep=},
    ]
    \addplot[const plot, blue, semithick] file{rysunki/PROJ1_PID_OPT_STERK=3.2526Ti=18.6051Td=0.01error=4.8389.txt};
    \legend{$u[k]$}
    \end{axis}
    \end{tikzpicture}
    \caption{Wyjscie dla $K_{\mathrm{r}}$=\num{3.25},  $T_{\mathrm{i}}$=\num{9.3}, $T_{\mathrm{d}}$=\num{0.005}}
    
\end{figure}


\begin{figure}[h]
    
    \textbf{E=\num{5.35}}
    \centering
    \begin{tikzpicture}
    \begin{axis}[
    width=0.98\textwidth,
    xmin=0,xmax=1200,ymin=0.4, ymax=1.2,
    xlabel={$k$},
    ylabel={$y[k]$},
    xtick={0, 300,600,900,1200},
    ytick={0.4,0.6,0.8,1.0,1.2},
    legend pos=south east,
    y tick label style={/pgf/number format/1000 sep=},
    ]
    \addplot[blue, semithick] file{rysunki/PROJ1_PID_OPTK=2.7409Ti=5.8109Td=21.9071error=5.3527.txt};
    \addplot[red, semithick] file{rysunki/PROJ1_4_Yzad.txt};
    \addlegendentry{$y[k]$},
    \addlegendentry{$y^{zad}[k]$},
    \addlegendimage{no markers, blue}
	\addlegendimage{no markers, red}
    \end{axis}
    \end{tikzpicture}
    \caption{Wyjscie dla $K_{\mathrm{r}}$=\num{2.74},  $T_{\mathrm{i}}$=\num{2.9}, $T_{\mathrm{d}}$=\num{10.95}}
    \label {PID_AUTO2}
\end{figure}

\begin{figure}[h]
    \centering
    \begin{tikzpicture}
    \begin{axis}[
    width=0.98\textwidth,
    xmin=0,xmax=1200,ymin=1.2, ymax=2.8,
    xlabel={$k$},
    ylabel={$u[k]$},
    xtick={0, 300,600,900,1200},
    ytick={1.2,1.6,2.0,2.4,2.8},
    legend pos=south east,
    y tick label style={/pgf/number format/1000 sep=},
    ]
    \addplot[const plot, blue, semithick] file{rysunki/PROJ1_PID_OPT_STERK=2.7409Ti=5.8109Td=21.9071error=5.3527.txt};
    \legend{$u[k]$}
    \end{axis}
    \end{tikzpicture}
    \caption{Sterowanie dla $K_{\mathrm{r}}$=\num{2.74},  $T_{\mathrm{i}}$=\num{2.9}, $T_{\mathrm{d}}$=\num{10.95} }
    
\end{figure}
\FloatBarrier

\section{Obserwacje i wnioski}

Ręczne dostrojenie regulatora PID okazało się odrobinę lepsze, być może z uwagi na bardzo małą niską wartoć parametru $T_{\mathrm{d}}$, leżącego poza obszarem poszukiwań algorytmu. Z reguły jednak niemożliwe jest tak dobre ręczne nastrojenie, bez użycia znaczącej ilosci czasu, automatyczny optymalizator gwarantuje praktycznie najlepsze rozwiązanie, jednak nie jest możliwe jego zastosowanie dla obiektów rzeczywistych poza symulacją, najwyżej co do wstępnej estymacji parametrów przy użyciu zebranego modelu obiektu.

