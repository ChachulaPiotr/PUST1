\chapter{Pomiar w punkcie pracy}
\section{Komunikacja z obiektem}
Komunikacja z obiektem odbywa si� za pomoc� funkcji napisanych w �rodowisku MatLab. Najprostszy program prezentuj�cy spos�b ich u�ycia znajduje si� poni�ej.
%\begin{lstlisting}[style=custommatlab,frame=single]
\begin{lstlisting}[style=Matlab-editor]
%Hello World in Matlab
addpath('F:\SerialCommunication'); % add a path to the functions
initSerialControl COM3 % initialise com port
N = 360;
i = 0;
while(i<=N)
	%% obtaining measurements
	measurements = readMeasurements(1:7); 
	
	%% processing of the measurements
	disp(measurements(1));
	
	%% sending new values of control signals
	sendControls([ 1, 2, 3, 4, 5, 6],
	[50, 0, 0, 0, 29, 0]);  % new corresponding control values
	
	i=i+1;
	% wait for new batch of measurements to be ready
	waitForNewIteration();
end
\end{lstlisting} 
Warto�ci podane do funkcji sendControls s� warto�ciami sterowania dla kolejno 4 wentylator�w i dw�ch grza�ek.
\section{Punkt pracy}
Sygna�em steruj�cym jest moc grzania pierwsz� grza�k� (5 element sterowalny obiektu), pierwszy wentylator (1 element sterowalny) ma by� na sta�e ustawiony na 50\% (cecha otoczenia), a pozosta�e elementy maj� pozosta� wy��czone. Wyj�ciem ma by� temperatura zmierzona przez czujnik temperatury T1.
Warto�ci� sygna�u steruj�cego w naszym punkcie pracy ma by� 29\%. Wielko�ci tej, po ustabilizowaniu si� obiektu, odpowiada temperatura na T1 r�wna \num{34.68}.