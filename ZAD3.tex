\chapter{Przekształcenie odpowiedzi skokowej}
\label{zad3}

\label{zad3_postac_ogolna}
Aby uzyskać znormalizowaną odpowiedź skokową, należy przerzutować ją względem punktu pracy oraz wielkosci skoku, a także przesuąć chwilę skoku sterowania do chwili k=\num{0} (z chwili $k_{skok}$).Do tego celu można użyć wzoru: 

\begin{equation}
    s_{i} = \frac{s_{i+k_{skok}} - Y_{\mathrm{pp}}}{\Delta U}
    \label{zad3_norm_odp_skok_wzor}
\end{equation}

Wyznaczono ją przy użyciu skrpytu \verb+PROJ1_3.m+ (dla odpowiedzi skokwej przy $\Delta u = \num{0,5}$). Następnie przycięto ją do miejsca w którym osiąga $\num{0.995}$ swojej maksymalnej wartosci. Długosc tej odpowiedzi jest przyjętym horyzontem dynamiki tego obiektu i jest równy 120. Wynik działania przedstawiony jest na rysunku \ref{zad3_norm_odp_wykres}.
Odpowiedź ta zostanie użyta do zaprojektowania regulatora DMC.

\begin{figure}[b]
    \label{zad3_norm_odp_wykres}
    \centering
    \begin{tikzpicture}
    \begin{axis}[
    width=\textwidth,
    xmin=0,xmax=120,ymin=0,ymax=0.8,
    xlabel={$k$},
    ylabel={$s_{k}$},
    xtick={0, 40, 80, 120},
    ytick={0, 0.2,0.4,0.6,0.8},
    legend pos=south east,
    y tick label style={/pgf/number format/1000 sep=},
    ]
    \addplot[red, semithick] file{rysunki/PROJ1_3.txt};
    \legend{$y[k]$}
    \end{axis}
    \end{tikzpicture}
    \caption{Postać przeksztalconej odpowiedzi skokowej symulowanego obiektu ze zmianą sterowania w momencie k=0}
\end{figure}




