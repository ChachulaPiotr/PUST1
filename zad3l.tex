\chapter{Skok jednostkowy}
\begin{center}
	% This file was created by matlab2tikz.
%
\definecolor{mycolor1}{rgb}{0.00000,0.44700,0.74100}%
%
\begin{tikzpicture}

\begin{axis}[%
width=4.521in,
height=3.566in,
at={(0.758in,0.481in)},
scale only axis,
xmin=0,
xmax=400,
xlabel style={font=\color{white!15!black}},
xlabel={[s]},
ymin=0,
ymax=0.35,
ylabel style={font=\color{white!15!black}},
ylabel={[�C]},
axis background/.style={fill=white},
title style={font=\bfseries},
title={Skok jednostkowy},
legend style={legend cell align=left, align=left, draw=white!15!black}
]
\addplot [color=mycolor1]
  table[row sep=crcr]{%
1	0\\
2	0\\
3	0\\
4	0\\
5	0.00700000000000003\\
6	0.00700000000000003\\
7	0.00700000000000003\\
8	0.00700000000000003\\
9	0.00700000000000003\\
10	0.00700000000000003\\
11	0.00700000000000003\\
12	0.00700000000000003\\
13	0.00700000000000003\\
14	0.0130000000000003\\
15	0.0130000000000003\\
16	0.0130000000000003\\
17	0.0189999999999998\\
18	0.0189999999999998\\
19	0.0189999999999998\\
20	0.0189999999999998\\
21	0.0189999999999998\\
22	0.0189999999999998\\
23	0.0189999999999998\\
24	0.0189999999999998\\
25	0.025\\
26	0.025\\
27	0.025\\
28	0.025\\
29	0.025\\
30	0.032\\
31	0.032\\
32	0.0380000000000003\\
33	0.0439999999999998\\
34	0.0439999999999998\\
35	0.05\\
36	0.057\\
37	0.057\\
38	0.0630000000000003\\
39	0.0630000000000003\\
40	0.0630000000000003\\
41	0.0689999999999998\\
42	0.075\\
43	0.082\\
44	0.082\\
45	0.082\\
46	0.0880000000000003\\
47	0.0880000000000003\\
48	0.0880000000000003\\
49	0.0880000000000003\\
50	0.0939999999999998\\
51	0.0939999999999998\\
52	0.0939999999999998\\
53	0.0939999999999998\\
54	0.0939999999999998\\
55	0.0939999999999998\\
56	0.1\\
57	0.1\\
58	0.107\\
59	0.107\\
60	0.113\\
61	0.119\\
62	0.119\\
63	0.119\\
64	0.119\\
65	0.119\\
66	0.125\\
67	0.125\\
68	0.132\\
69	0.132\\
70	0.138\\
71	0.144\\
72	0.144\\
73	0.15\\
74	0.15\\
75	0.15\\
76	0.157\\
77	0.157\\
78	0.163\\
79	0.163\\
80	0.163\\
81	0.169\\
82	0.169\\
83	0.169\\
84	0.169\\
85	0.175\\
86	0.175\\
87	0.175\\
88	0.175\\
89	0.175\\
90	0.175\\
91	0.175\\
92	0.182\\
93	0.188\\
94	0.188\\
95	0.194\\
96	0.194\\
97	0.194\\
98	0.194\\
99	0.194\\
100	0.194\\
101	0.2\\
102	0.2\\
103	0.2\\
104	0.2\\
105	0.207\\
106	0.207\\
107	0.207\\
108	0.207\\
109	0.207\\
110	0.207\\
111	0.213\\
112	0.213\\
113	0.213\\
114	0.213\\
115	0.213\\
116	0.213\\
117	0.219\\
118	0.213\\
119	0.219\\
120	0.219\\
121	0.219\\
122	0.225\\
123	0.225\\
124	0.225\\
125	0.232\\
126	0.238\\
127	0.238\\
128	0.238\\
129	0.232\\
130	0.232\\
131	0.232\\
132	0.238\\
133	0.238\\
134	0.232\\
135	0.232\\
136	0.232\\
137	0.238\\
138	0.238\\
139	0.244\\
140	0.25\\
141	0.25\\
142	0.25\\
143	0.244\\
144	0.244\\
145	0.244\\
146	0.244\\
147	0.244\\
148	0.244\\
149	0.244\\
150	0.244\\
151	0.25\\
152	0.25\\
153	0.25\\
154	0.257\\
155	0.257\\
156	0.257\\
157	0.257\\
158	0.257\\
159	0.257\\
160	0.257\\
161	0.25\\
162	0.25\\
163	0.257\\
164	0.25\\
165	0.257\\
166	0.257\\
167	0.257\\
168	0.263\\
169	0.257\\
170	0.263\\
171	0.263\\
172	0.263\\
173	0.263\\
174	0.263\\
175	0.263\\
176	0.263\\
177	0.263\\
178	0.263\\
179	0.263\\
180	0.263\\
181	0.263\\
182	0.257\\
183	0.257\\
184	0.263\\
185	0.257\\
186	0.257\\
187	0.257\\
188	0.25\\
189	0.25\\
190	0.25\\
191	0.25\\
192	0.25\\
193	0.25\\
194	0.257\\
195	0.25\\
196	0.257\\
197	0.263\\
198	0.263\\
199	0.263\\
200	0.263\\
201	0.257\\
202	0.257\\
203	0.257\\
204	0.257\\
205	0.257\\
206	0.257\\
207	0.263\\
208	0.263\\
209	0.263\\
210	0.263\\
211	0.263\\
212	0.263\\
213	0.263\\
214	0.263\\
215	0.263\\
216	0.263\\
217	0.263\\
218	0.263\\
219	0.263\\
220	0.263\\
221	0.257\\
222	0.257\\
223	0.257\\
224	0.263\\
225	0.263\\
226	0.263\\
227	0.263\\
228	0.263\\
229	0.263\\
230	0.263\\
231	0.263\\
232	0.263\\
233	0.263\\
234	0.269\\
235	0.263\\
236	0.269\\
237	0.263\\
238	0.269\\
239	0.269\\
240	0.269\\
241	0.269\\
242	0.269\\
243	0.269\\
244	0.275\\
245	0.275\\
246	0.275\\
247	0.275\\
248	0.275\\
249	0.282\\
250	0.282\\
251	0.282\\
252	0.282\\
253	0.288\\
254	0.282\\
255	0.282\\
256	0.282\\
257	0.275\\
258	0.275\\
259	0.275\\
260	0.275\\
261	0.275\\
262	0.282\\
263	0.282\\
264	0.288\\
265	0.282\\
266	0.282\\
267	0.282\\
268	0.282\\
269	0.282\\
270	0.282\\
271	0.288\\
272	0.282\\
273	0.282\\
274	0.282\\
275	0.282\\
276	0.282\\
277	0.282\\
278	0.282\\
279	0.282\\
280	0.282\\
281	0.282\\
282	0.282\\
283	0.282\\
284	0.282\\
285	0.282\\
286	0.282\\
287	0.288\\
288	0.288\\
289	0.288\\
290	0.288\\
291	0.288\\
292	0.288\\
293	0.282\\
294	0.282\\
295	0.288\\
296	0.288\\
297	0.288\\
298	0.288\\
299	0.288\\
300	0.288\\
301	0.288\\
302	0.288\\
303	0.282\\
304	0.282\\
305	0.282\\
306	0.282\\
307	0.282\\
308	0.282\\
309	0.282\\
310	0.282\\
311	0.288\\
312	0.288\\
313	0.288\\
314	0.288\\
315	0.288\\
316	0.288\\
317	0.288\\
318	0.288\\
319	0.288\\
320	0.288\\
321	0.294\\
322	0.294\\
323	0.3\\
324	0.3\\
325	0.3\\
326	0.3\\
327	0.3\\
328	0.294\\
329	0.294\\
330	0.3\\
331	0.3\\
332	0.307\\
333	0.307\\
334	0.307\\
335	0.307\\
336	0.307\\
337	0.3\\
338	0.307\\
339	0.3\\
340	0.3\\
341	0.294\\
342	0.294\\
343	0.3\\
344	0.3\\
345	0.294\\
346	0.3\\
347	0.3\\
348	0.3\\
349	0.3\\
350	0.3\\
351	0.3\\
352	0.307\\
353	0.307\\
354	0.313\\
355	0.313\\
356	0.313\\
357	0.313\\
358	0.313\\
359	0.307\\
360	0.307\\
361	0.307\\
};
\addlegendentry{skok jednostkowy}

\end{axis}
\end{tikzpicture}%
\end{center}
Do otrzymania odpowiedzi skokowej wykorzystywanej w algorytmie DMC u�yli�my pomiar�w uzyskanych przy skoku warto�ci sterowania o 40. Nast�pnie dokonali�my aproksymacji uzyskanej odpowiedzi u�ywaj�c cz�onu inercyjnego drugiego rz�du z op�nieniem. 

\begin{center}
	% This file was created by matlab2tikz.
%
\definecolor{mycolor1}{rgb}{0.00000,0.44700,0.74100}%
\definecolor{mycolor2}{rgb}{0.85000,0.32500,0.09800}%
%
\begin{tikzpicture}

\begin{axis}[%
width=4.521in,
height=3.566in,
at={(0.758in,0.481in)},
scale only axis,
xmin=0,
xmax=400,
xlabel style={font=\color{white!15!black}},
xlabel={[s]},
ymin=0,
ymax=0.3,
ylabel style={font=\color{white!15!black}},
ylabel={[�C]},
axis background/.style={fill=white},
title style={font=\bfseries},
title={Skok jednostkowy},
legend style={at={(0.615,0.743)}, anchor=south west, legend cell align=left, align=left, draw=white!15!black}
]
\addplot [color=mycolor1]
  table[row sep=crcr]{%
1	0\\
2	0\\
3	0\\
4	0\\
5	0\\
6	0\\
7	0\\
8	0\\
9	0.00150000000000006\\
10	0.00150000000000006\\
11	0.00150000000000006\\
12	0.00299999999999994\\
13	0.00449999999999999\\
14	0.00449999999999999\\
15	0.00625\\
16	0.00775000000000006\\
17	0.00924999999999994\\
18	0.00924999999999994\\
19	0.01075\\
20	0.0125\\
21	0.0140000000000001\\
22	0.0154999999999999\\
23	0.017\\
24	0.01875\\
25	0.0217499999999999\\
26	0.02325\\
27	0.025\\
28	0.0279999999999999\\
29	0.03125\\
30	0.0327500000000001\\
31	0.03575\\
32	0.0375\\
33	0.0404999999999999\\
34	0.042\\
35	0.0452500000000001\\
36	0.04825\\
37	0.0515000000000001\\
38	0.0529999999999999\\
39	0.05625\\
40	0.0592499999999999\\
41	0.06075\\
42	0.0640000000000001\\
43	0.067\\
44	0.06875\\
45	0.0717499999999999\\
46	0.075\\
47	0.0765000000000001\\
48	0.0795\\
49	0.0827500000000001\\
50	0.08575\\
51	0.0890000000000001\\
52	0.092\\
53	0.0952500000000001\\
54	0.0967499999999999\\
55	0.1\\
56	0.1015\\
57	0.1045\\
58	0.10775\\
59	0.11075\\
60	0.1125\\
61	0.1155\\
62	0.117\\
63	0.12025\\
64	0.12175\\
65	0.125\\
66	0.128\\
67	0.1295\\
68	0.13275\\
69	0.13425\\
70	0.1375\\
71	0.139\\
72	0.1405\\
73	0.142\\
74	0.14525\\
75	0.14675\\
76	0.14825\\
77	0.15\\
78	0.1515\\
79	0.153\\
80	0.1545\\
81	0.15775\\
82	0.15925\\
83	0.1625\\
84	0.164\\
85	0.1655\\
86	0.167\\
87	0.17025\\
88	0.17325\\
89	0.175\\
90	0.178\\
91	0.178\\
92	0.18125\\
93	0.18275\\
94	0.18425\\
95	0.18575\\
96	0.1875\\
97	0.189\\
98	0.1905\\
99	0.192\\
100	0.19375\\
101	0.19525\\
102	0.19675\\
103	0.19825\\
104	0.2\\
105	0.2015\\
106	0.203\\
107	0.20625\\
108	0.20625\\
109	0.20775\\
110	0.20925\\
111	0.21075\\
112	0.2125\\
113	0.214\\
114	0.2155\\
115	0.217\\
116	0.217\\
117	0.21875\\
118	0.21875\\
119	0.22025\\
120	0.22025\\
121	0.22175\\
122	0.22175\\
123	0.22325\\
124	0.225\\
125	0.22325\\
126	0.225\\
127	0.225\\
128	0.225\\
129	0.2265\\
130	0.228\\
131	0.228\\
132	0.228\\
133	0.2295\\
134	0.23125\\
135	0.23275\\
136	0.23425\\
137	0.23425\\
138	0.23575\\
139	0.23575\\
140	0.23425\\
141	0.23575\\
142	0.23575\\
143	0.2375\\
144	0.239\\
145	0.239\\
146	0.239\\
147	0.239\\
148	0.239\\
149	0.2405\\
150	0.242\\
151	0.242\\
152	0.24375\\
153	0.24375\\
154	0.24525\\
155	0.24525\\
156	0.24525\\
157	0.24525\\
158	0.24525\\
159	0.24525\\
160	0.24675\\
161	0.24675\\
162	0.24825\\
163	0.24825\\
164	0.25\\
165	0.2515\\
166	0.2515\\
167	0.253\\
168	0.253\\
169	0.2545\\
170	0.25625\\
171	0.25625\\
172	0.25625\\
173	0.25775\\
174	0.25775\\
175	0.25925\\
176	0.26075\\
177	0.26075\\
178	0.2625\\
179	0.2625\\
180	0.2625\\
181	0.264\\
182	0.264\\
183	0.2655\\
184	0.2655\\
185	0.2655\\
186	0.2655\\
187	0.2655\\
188	0.2655\\
189	0.2655\\
190	0.2655\\
191	0.2655\\
192	0.2655\\
193	0.267\\
194	0.267\\
195	0.26875\\
196	0.26875\\
197	0.26875\\
198	0.26875\\
199	0.26875\\
200	0.26875\\
201	0.26875\\
202	0.26875\\
203	0.27025\\
204	0.26875\\
205	0.267\\
206	0.267\\
207	0.267\\
208	0.267\\
209	0.26875\\
210	0.27025\\
211	0.27025\\
212	0.27025\\
213	0.27025\\
214	0.27025\\
215	0.27025\\
216	0.27025\\
217	0.27175\\
218	0.27025\\
219	0.27025\\
220	0.27025\\
221	0.27025\\
222	0.27175\\
223	0.27175\\
224	0.27325\\
225	0.27325\\
226	0.27325\\
227	0.27325\\
228	0.27325\\
229	0.27325\\
230	0.275\\
231	0.275\\
232	0.2765\\
233	0.278\\
234	0.278\\
235	0.2795\\
236	0.2795\\
237	0.28125\\
238	0.28275\\
239	0.28275\\
240	0.28425\\
241	0.28575\\
242	0.28575\\
243	0.28575\\
244	0.28575\\
245	0.28425\\
246	0.28425\\
247	0.28575\\
248	0.28575\\
249	0.2875\\
250	0.2875\\
251	0.2875\\
252	0.289\\
253	0.289\\
254	0.2875\\
255	0.2875\\
256	0.2875\\
257	0.28575\\
258	0.28575\\
259	0.28575\\
260	0.28575\\
261	0.28575\\
262	0.28425\\
263	0.28425\\
264	0.28425\\
265	0.28575\\
266	0.28425\\
267	0.28425\\
268	0.28425\\
269	0.28425\\
270	0.28275\\
271	0.28125\\
272	0.2795\\
273	0.2795\\
274	0.2795\\
275	0.278\\
276	0.278\\
277	0.278\\
278	0.278\\
279	0.2795\\
280	0.28125\\
281	0.28275\\
282	0.28425\\
283	0.28425\\
284	0.28425\\
285	0.28425\\
286	0.28425\\
287	0.28275\\
288	0.28275\\
289	0.28425\\
290	0.28425\\
291	0.28425\\
292	0.28575\\
293	0.28575\\
294	0.28425\\
295	0.28425\\
296	0.28425\\
297	0.28575\\
298	0.28575\\
299	0.28575\\
300	0.28575\\
301	0.2875\\
302	0.28575\\
303	0.28575\\
304	0.28575\\
305	0.28425\\
306	0.28425\\
307	0.28275\\
308	0.28275\\
309	0.28125\\
310	0.28125\\
311	0.28275\\
312	0.28275\\
313	0.28275\\
314	0.28275\\
315	0.28275\\
316	0.28275\\
317	0.28275\\
318	0.28125\\
319	0.28125\\
320	0.28125\\
321	0.28125\\
322	0.2795\\
323	0.2795\\
324	0.2795\\
325	0.2795\\
326	0.2795\\
327	0.278\\
328	0.2795\\
329	0.2795\\
330	0.2795\\
331	0.2795\\
332	0.28125\\
333	0.28275\\
334	0.28275\\
335	0.28275\\
336	0.28275\\
337	0.28425\\
338	0.28425\\
339	0.28425\\
340	0.28575\\
341	0.28575\\
342	0.28575\\
343	0.2875\\
344	0.2875\\
345	0.2875\\
346	0.289\\
347	0.289\\
348	0.289\\
349	0.289\\
350	0.289\\
351	0.2905\\
352	0.2905\\
353	0.2905\\
354	0.289\\
355	0.289\\
356	0.289\\
357	0.289\\
358	0.289\\
359	0.289\\
360	0.289\\
361	0.289\\
};
\addlegendentry{skok jednostkowy}

\addplot [color=mycolor2]
  table[row sep=crcr]{%
1	0\\
2	0\\
3	0\\
4	7.79937924069266e-05\\
5	0.00030713059222033\\
6	0.000680333896144581\\
7	0.00119077471460944\\
8	0.00183186388364738\\
9	0.0025972446005736\\
10	0.00348078517721513\\
11	0.00447657200460555\\
12	0.00557890272322705\\
13	0.00678227959304254\\
14	0.00808140305771718\\
15	0.00947116549758116\\
16	0.0109466451660341\\
17	0.0125031003042358\\
18	0.0141359634290685\\
19	0.015840835789493\\
20	0.0176134819865547\\
21	0.0194498247524229\\
22	0.0213459398839769\\
23	0.0232980513265706\\
24	0.0253025264037318\\
25	0.0273558711886629\\
26	0.0294547260135291\\
27	0.0315958611126262\\
28	0.0337761723956269\\
29	0.0359926773472139\\
30	0.0382425110495031\\
31	0.0405229223237632\\
32	0.0428312699880338\\
33	0.0451650192273357\\
34	0.0475217380732606\\
35	0.0498990939898139\\
36	0.0522948505624718\\
37	0.0547068642874983\\
38	0.057133081458647\\
39	0.0595715351484552\\
40	0.0620203422814124\\
41	0.0644777007963623\\
42	0.0669418868955696\\
43	0.0694112523779549\\
44	0.0718842220540692\\
45	0.0743592912404481\\
46	0.0768350233310511\\
47	0.0793100474435541\\
48	0.0817830561383275\\
49	0.0842528032079899\\
50	0.0867181015354896\\
51	0.0891778210187193\\
52	0.0916308865597286\\
53	0.0940762761166515\\
54	0.0965130188165175\\
55	0.0989401931271691\\
56	0.101356925086555\\
57	0.103762386587721\\
58	0.106155793717861\\
59	0.108536405149844\\
60	0.110903520584674\\
61	0.11325647924338\\
62	0.115594658406886\\
63	0.117917472002431\\
64	0.120224369235184\\
65	0.122514833263688\\
66	0.124788379917868\\
67	0.127044556458301\\
68	0.12928294037556\\
69	0.13150313822841\\
70	0.133704784519705\\
71	0.135887540608876\\
72	0.138051093659892\\
73	0.140195155623647\\
74	0.142319462253736\\
75	0.144423772154618\\
76	0.146507865861191\\
77	0.148571544948833\\
78	0.150614631172983\\
79	0.152636965637384\\
80	0.154638407990106\\
81	0.15661883564651\\
82	0.158578143038334\\
83	0.160516240888117\\
84	0.162433055508166\\
85	0.164328528123341\\
86	0.166202614216907\\
87	0.168055282898764\\
88	0.169886516295361\\
89	0.171696308960617\\
90	0.173484667307221\\
91	0.175251609057668\\
92	0.176997162714418\\
93	0.178721367048599\\
94	0.180424270606666\\
95	0.182105931234461\\
96	0.183766415618128\\
97	0.185405798841366\\
98	0.187024163958496\\
99	0.188621601582853\\
100	0.190198209490018\\
101	0.191754092235425\\
102	0.193289360785887\\
103	0.194804132164598\\
104	0.196298529109189\\
105	0.197772679742414\\
106	0.199226717255076\\
107	0.200660779600787\\
108	0.202075009202194\\
109	0.203469552668301\\
110	0.204844560522521\\
111	0.20620018694112\\
112	0.207536589501724\\
113	0.208853928941534\\
114	0.210152368924975\\
115	0.211432075820427\\
116	0.212693218485776\\
117	0.213935968062479\\
118	0.215160497777859\\
119	0.216366982755374\\
120	0.21755559983258\\
121	0.218726527386545\\
122	0.219879945166455\\
123	0.221016034133177\\
124	0.222134976305557\\
125	0.223236954613197\\
126	0.224322152755529\\
127	0.225390755066939\\
128	0.226442946387766\\
129	0.227478911940951\\
130	0.228498837214152\\
131	0.229502907847149\\
132	0.230491309524336\\
133	0.231464227872135\\
134	0.232421848361166\\
135	0.233364356212995\\
136	0.234291936311313\\
137	0.235204773117374\\
138	0.236103050589564\\
139	0.236986952106929\\
140	0.237856660396545\\
141	0.238712357464577\\
142	0.239554224530899\\
143	0.240382441967147\\
144	0.241197189238086\\
145	0.241998644846155\\
146	0.242786986279097\\
147	0.243562389960536\\
148	0.244325031203405\\
149	0.245075084166129\\
150	0.245812721811431\\
151	0.246538115867696\\
152	0.247251436792768\\
153	0.247952853740111\\
154	0.248642534527226\\
155	0.249320645606245\\
156	0.249987352036616\\
157	0.250642817459806\\
158	0.251287204075924\\
159	0.251920672622208\\
160	0.252543382353289\\
161	0.253155491023169\\
162	0.253757154868839\\
163	0.254348528595469\\
164	0.254929765363115\\
165	0.255501016774869\\
166	0.256062432866396\\
167	0.256614162096806\\
168	0.257156351340796\\
169	0.257689145882012\\
170	0.258212689407577\\
171	0.258727124003738\\
172	0.259232590152574\\
173	0.259729226729732\\
174	0.260217171003128\\
175	0.260696558632589\\
176	0.26116752367037\\
177	0.261630198562526\\
178	0.262084714151089\\
179	0.262531199677009\\
180	0.262969782783834\\
181	0.263400589522079\\
182	0.26382374435426\\
183	0.264239370160552\\
184	0.264647588245052\\
185	0.265048518342598\\
186	0.265442278626128\\
187	0.265828985714548\\
188	0.266208754681074\\
189	0.266581699062034\\
190	0.26694793086609\\
191	0.267307560583872\\
192	0.267660697197978\\
193	0.268007448193343\\
194	0.268347919567937\\
195	0.26868221584377\\
196	0.269010440078203\\
197	0.269332693875517\\
198	0.26964907739875\\
199	0.26995968938176\\
200	0.270264627141508\\
201	0.270563986590553\\
202	0.270857862249718\\
203	0.271146347260937\\
204	0.271429533400254\\
205	0.271707511090965\\
206	0.271980369416887\\
207	0.27224819613574\\
208	0.272511077692634\\
209	0.272769099233649\\
210	0.273022344619485\\
211	0.273270896439188\\
212	0.273514836023929\\
213	0.273754243460835\\
214	0.273989197606849\\
215	0.274219776102625\\
216	0.274446055386445\\
217	0.274668110708135\\
218	0.274886016142997\\
219	0.27509984460573\\
220	0.27530966786434\\
221	0.275515556554028\\
222	0.275717580191059\\
223	0.275915807186594\\
224	0.276110304860485\\
225	0.276301139455027\\
226	0.276488376148659\\
227	0.276672079069612\\
228	0.276852311309503\\
229	0.27702913493685\\
230	0.277202611010541\\
231	0.277372799593206\\
232	0.277539759764536\\
233	0.277703549634505\\
234	0.277864226356514\\
235	0.278021846140453\\
236	0.278176464265666\\
237	0.27832813509383\\
238	0.278476912081728\\
239	0.27862284779394\\
240	0.278765993915417\\
241	0.278906401263965\\
242	0.279044119802617\\
243	0.279179198651903\\
244	0.279311686102009\\
245	0.279441629624827\\
246	0.279569075885896\\
247	0.279694070756225\\
248	0.279816659324009\\
249	0.279936885906222\\
250	0.280054794060106\\
251	0.280170426594527\\
252	0.280283825581229\\
253	0.280395032365961\\
254	0.280504087579484\\
255	0.280611031148468\\
256	0.280715902306258\\
257	0.280818739603524\\
258	0.280919580918796\\
259	0.281018463468874\\
260	0.281115423819111\\
261	0.28121049789359\\
262	0.281303720985167\\
263	0.281395127765401\\
264	0.28148475229436\\
265	0.281572628030311\\
266	0.281658787839287\\
267	0.281743264004532\\
268	0.281826088235835\\
269	0.281907291678737\\
270	0.281986904923625\\
271	0.282064958014705\\
272	0.282141480458863\\
273	0.282216501234402\\
274	0.282290048799669\\
275	0.282362151101564\\
276	0.282432835583937\\
277	0.282502129195868\\
278	0.282570058399837\\
279	0.282636649179776\\
280	0.28270192704902\\
281	0.282765917058136\\
282	0.282828643802649\\
283	0.282890131430657\\
284	0.282950403650339\\
285	0.283009483737353\\
286	0.283067394542133\\
287	0.283124158497073\\
288	0.283179797623618\\
289	0.283234333539237\\
290	0.283287787464306\\
291	0.283340180228885\\
292	0.283391532279393\\
293	0.283441863685187\\
294	0.283491194145039\\
295	0.283539542993517\\
296	0.283586929207274\\
297	0.283633371411233\\
298	0.283678887884684\\
299	0.283723496567287\\
300	0.283767215064984\\
301	0.283810060655811\\
302	0.283852050295635\\
303	0.283893200623787\\
304	0.283933527968616\\
305	0.283973048352957\\
306	0.284011777499501\\
307	0.284049730836099\\
308	0.284086923500963\\
309	0.284123370347799\\
310	0.284159085950849\\
311	0.284194084609855\\
312	0.284228380354949\\
313	0.284261986951449\\
314	0.284294917904598\\
315	0.284327186464209\\
316	0.284358805629242\\
317	0.284389788152306\\
318	0.284420146544087\\
319	0.2844498930777\\
320	0.284479039792978\\
321	0.284507598500678\\
322	0.284535580786627\\
323	0.284562998015795\\
324	0.284589861336302\\
325	0.284616181683356\\
326	0.284641969783129\\
327	0.28466723615656\\
328	0.284691991123106\\
329	0.284716244804416\\
330	0.284740007127955\\
331	0.284763287830562\\
332	0.284786096461941\\
333	0.284808442388106\\
334	0.284830334794754\\
335	0.28485178269059\\
336	0.284872794910591\\
337	0.284893380119208\\
338	0.284913546813529\\
339	0.284933303326367\\
340	0.284952657829314\\
341	0.284971618335726\\
342	0.284990192703666\\
343	0.285008388638794\\
344	0.285026213697202\\
345	0.285043675288208\\
346	0.285060780677091\\
347	0.285077536987787\\
348	0.28509395120553\\
349	0.285110030179455\\
350	0.285125780625145\\
351	0.285141209127142\\
352	0.28515632214141\\
353	0.285171125997751\\
354	0.285185626902184\\
355	0.285199830939279\\
356	0.285213744074446\\
357	0.285227372156192\\
358	0.285240720918327\\
359	0.285253795982139\\
360	0.285266602858523\\
361	0.285279146950078\\
};
\addlegendentry{aproksymacja}

\end{axis}
\end{tikzpicture}%
\end{center}
\begin{equation}
G(s)=\frac{\num{0.2859}}{(\num{42}s+1)(\num{43}+1)}e^{-3s}
\nonumber
\end{equation}
Wynik taki otrzymali�my korzystaj�c z funkcji fmincon d���c do minimalizacji b��du �redniokwadratowego aproksymacji.