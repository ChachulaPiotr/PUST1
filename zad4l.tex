\chapter{Regulatory}
\section{PID}
\begin{lstlisting}[style=Matlab-editor]
addpath('F:\SerialCommunication'); % add a path to the functions
initSerialControl COM3 % initialise com port

K=38;
Ti=1/12;
Td=3;
T=1; % Czas probkowania
error=0;
sim_len=600;

% Parametry dyskretnego PIDa
r0=K*(1+T/(2*Ti)+Td/T);
r1=K*(T/(2*Ti)-2*Td/T-1);
r2=K*Td/T;

Y=zeros(sim_len,1);
U=zeros(sim_len,1);
e=zeros(sim_len,1);
y=zeros(sim_len,1);
u=zeros(sim_len,1);
Yzad=zeros(sim_len,1);

Ypp=34.5;
Upp=29;

Y(1:30)=Ypp;
U(1:30)=Upp;
kk=linspace(1,sim_len,sim_len)';


Yzad(1:sim_len/3-1)=37.0;
Yzad(sim_len/3:2*sim_len/3-1)=33.0;
Yzad(2*sim_len/3:sim_len)=34.5;


% Ograniczenia U
Umin=0;
Umax=100;
umin=Umin-Upp;
umax=Umax-Upp;

for i=31:sim_len
measurements = readMeasurements(1:7); % read measurements

Y(k)=measurements(1)
y(k)=Y(k)-Ypp;
e(k)=Yzad(k)-Y(k);
error=error+e(k)^2;

u_wyliczone=r2*e(k-2)+r1*e(k-1)+r0*e(k)+u(k-1);
% Rzutowanie ograniczen na wartosc sterowania
if u_wyliczone<umin
u_wyliczone=umin;
elseif u_wyliczone>umax
u_wyliczone=umax;
end
u(k)=u_wyliczone;
U(k)=u_wyliczone+Upp;
sendControls([ 1, 2, 3, 4, 5, 6], ... send for these elements
[50, 0, 0, 0, U(k), 0]);  % new corresponding control values
end
\end{lstlisting} 
\section{DMC}
\begin{lstlisting}[style=Matlab-editor]
load normal_step_response.mat;
% Ucinamy moment 0 zeby latwiej bylo operowac
stepResp=stepResp(2:end);

D=length(step_response);
N=80;
Nu=60;
lambda=1;

addpath('F:\SerialCommunication'); % add a path to the functions
initSerialControl COM3 % initialise com port

error=0;

M=zeros(N,Nu);
for j=1:Nu
	for i=j:N
		M(i,j)=stepResp(i-j+1);
	end
end

Mp=zeros(N,D-1);
for i=1:N
	for j=1:D-1
		if (i+j)<=D
			Mp(i,j)=stepResp(i+j)-stepResp(j);
		else
			Mp(i,j)=stepResp(end)-stepResp(j);
		end
	end
end


K=(M'*M+lambda*lambda*eye(Nu))^(-1)*M';
Ke=sum(K(1,:));
Ku=K(1,:)*Mp;

dUp=zeros(D-1,1);

% Koniec czesci przedregulacyjnej

sim_len=1800;

Y=zeros(sim_len,1);
U=zeros(sim_len,1);
du=zeros(sim_len,1);
e=zeros(sim_len,1);
y=zeros(sim_len,1);
u=zeros(sim_len,1);
Yzad=zeros(sim_len,1);


Yzad(1:sim_len/3-1)=37.0;
Yzad(sim_len/3:2*sim_len/3-1)=33.0;
Yzad(2*sim_len/3:sim_len)=34.5;


Ypp=34.5;
Upp=29.0;

Y(1:D+20)=0.8;
U(1:D+20)=2.0;

kk=linspace(1,sim_len,sim_len)';

Umin=1.2;
Umax=2.8;
umin=Umin-Upp;
umax=Umax-Upp;

% Koniec deklaracji innych waznych rzeczy

for k=D+21:sim_len
	measurements = readMeasurements(1:7); % read measurements
	Y(k)=measurements(1)
	y(k)=Y(k)-Ypp;
	e(k)=Yzad(k)-Y(k);
	error=error+e(k)^2;
	
	dUp=du(k-D+1:k-1);
	du_wyliczone=Ke*e(k)-Ku*dUp;
	
	% Rzutowanie ograniczen na wartosc zmiany sterowania
	if du_wyliczone+u(k-1)<umin
		du_wyliczone=umin-u(k-1);
	elseif du_wyliczone+u(k-1)>umax
		du_wyliczone=umax-u(k-1);
	end
	
	du(k)=du_wyliczone;
	u(k)=u(k-1)+du(k);
	U(k)=u(k)+Upp;
	sendControls([ 1, 2, 3, 4, 5, 6], ... send for these elements
	[50, 0, 0, 0, U(k), 0]);  % new corresponding control values
end
\end{lstlisting} 