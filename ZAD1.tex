\chapter{Weryfikacja punktu pracy}
\label{zad1}

\section{Opis postępowania}
\label{zad1_opis}
W celu sprawdzenia poprawności wartości sygnałów $U_{\mathrm{pp}}$ i $Y_{\mathrm{pp}}$ pobudzono obiekt sterowaniem
o wartości $U_{\mathrm{pp}}= \num{2.0}$ i sprawdzeniu czy stabilizuje się on w punkcjie pracy  $Y_{\mathrm{pp}}= \num{0.8}$. Do symulacji wyjscia obiektu użyto udostępnionej funkcji 
\verb+symulacja_obiektu4Y.+ Do testów napisano skrypt \verb+PROJ1_1.m. + Wyniki przedstawiono poniżej.

\section{Wyniki}
\label{zad1_wyniki}
Zgodnie z przewidywaniami wyjscie obiektu ustaliło się na wartości $Y_{\mathrm{pp}}= \num{2.0}$. Punkt pracy ustalony jest więc poprawnie.
\begin{figure}[b]

    \label{zad1_1_wykres}
    \centering
    \begin{tikzpicture}
    \begin{axis}[
    width=\textwidth,
    xmin=0,xmax=200,ymin=0,ymax=0.8,
    xlabel={$k$},
    ylabel={$y[k]$},
    xtick={0, 50, 100, 150, 200},
    ytick={0, 0.2, 0.4, 0.6, 0.8},
    legend pos=south east,
    y tick label style={/pgf/number format/1000 sep=},
    ]
    \addplot[red, semithick] file{rysunki/PROJ1_1.txt};
    \legend{$y[k]$}
    \end{axis} 
    \end{tikzpicture}
    \caption{Odpowiedź obiektu na sterowaniei $U_{\mathrm{pp}}=\num{0.8}$}
\end{figure}