\chapter{Regulator DMC}
\label{zad5}


\section{Algorytm działania}
Algorytm działania regulatora oraz implementacja została dobrze udokumentowana w pliku \verb+doDMC.m +. Listing jego częsci algorytmicznej przedstawiony jest poniżej:
\begin{lstlisting}[style=custommatlab,frame=single,label={zad4_sim_lst},caption={Implementacja regulatora DMC},captionpos=b]

function [ error ] = doDMC( paras )   

% Zaokrgalamy wartosci dla algorytmow optymalizacji
N=round(paras(1));
Nu=round(paras(2));
lambda=paras(3);

% Zaladowanie odpowiedzi skokowej obiektu
load StepResponse.mat;

% Ucinamy moment 0 zeby latwiej bylo operowac
stepResp=stepResp(2:end);

% Pobieramy horyzont dynamiki obiektu
D=length(stepResp);

% Inicjalizacja wspolczynnika jakosci
error=0;

% Inicjalizacja macierzy M
M=zeros(N,Nu);
for j=1:Nu
    for i=j:N
        M(i,j)=stepResp(i-j+1);
    end
end

% Inicjalizacja macierzy Mp
Mp=zeros(N,D-1);
for i=1:N
    for j=1:D-1
        if (i+j)<=D
            Mp(i,j)=stepResp(i+j)-stepResp(j);
        else
            Mp(i,j)=stepResp(end)-stepResp(j);
        end
    end
end

% Liczymy macierze K, Ke, Ku
K=(M'*M+lambda*lambda*eye(Nu))^(-1)*M';
Ke=sum(K(1,:));
Ku=K(1,:)*Mp;

% Inicjalizujemy macierze przechowujace zmienne

sim_len=1200;
dUp=zeros(D-1,1);
Y=zeros(sim_len,1);
U=zeros(sim_len,1);
du=zeros(sim_len,1);
e=zeros(sim_len,1);
y=zeros(sim_len,1);
u=zeros(sim_len,1);
Yzad=zeros(sim_len,1);
kk=linspace(1,sim_len,sim_len)';

% Tworzymy horyzont wartosci zadanej
Yzad(1:D+11)=0.8;
Yzad(D+12:sim_len/3-1)=1.0;
Yzad(sim_len/3:2*sim_len/3-1)=0.6;
Yzad(2*sim_len/3:sim_len)=0.7;

% Ustalamy wartosci przed rozpoczeciem symulacji na wartosci w punktu pracy
Ypp=0.8;
Upp=2.0;
Y(1:D+11)=0.8;
U(1:D+11)=2.0;

% Wprowadzamy ograniczenia
Umin=1.2;
Umax=2.8;
deltaumax=0.25;
deltaumin=-0.25;
umin=Umin-Upp;
umax=Umax-Upp;

% Poczatek symulacji - zaczynamy w tej chwili w celu uproszczenia
% pozyskiwania wektora dUp
for k=D+12:sim_len
    % Symulujemy wyjscie obiektu
    Y(k)=symulacja_obiektu4Y(U(k-10),U(k-11),Y(k-1),Y(k-2));
    % Rzutujemy wartosc wyjscia wzgledem punktu pracy
    y(k)=Y(k)-Ypp;
    % Liczymy uchyb i uaktualniamy wspolczynnik bledu
    e(k)=Yzad(k)-Y(k);
    error=error+e(k)^2;
    % Pozyskujemy wektor dUp z wektora du
    dUp=du(k-D+1:k-1);
    dUp=flip(dUp);
    % Liczymy wartosc zmiany sterowania
    du_wyliczone=Ke*e(k)-Ku*dUp;

    % Rzutowanie ograniczen na wartosc sterowania
    if du_wyliczone<deltaumin
        du_wyliczone=deltaumin;
    elseif du_wyliczone>deltaumax
        du_wyliczone=deltaumax;
    end
    % Rzutowanie ograniczen na wartosc zmiany sterowania
    if du_wyliczone+u(k-1)<umin
        du_wyliczone=umin-u(k-1);
    elseif du_wyliczone+u(k-1)>umax
        du_wyliczone=umax-u(k-1);
    end
    du(k)=du_wyliczone;
    % Liczymy wartosc sterowania i ja rzutujemy wzgledem punktu pracy
    u(k)=u(k-1)+du(k);
    U(k)=u(k)+Upp;
end
    


\end{lstlisting}
