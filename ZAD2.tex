\chapter{Odpowiedzi skokowe}
\label{zad2}

\section{Wyznaczanie odpowiedzi skokwych}
\label{zad2_skoki}

\begin{figure}[t]
\label{zad2_skoki_wykres}
    \centering
    \begin{tikzpicture}
    \begin{axis}[
    width=0.98\textwidth,
    xmin=0.0,xmax=200,ymin=0.8,ymax=1.2,
    xlabel={$k$},
    ylabel={$y[k]$},
    legend pos=south east,
    y tick label style={/pgf/number format/1000 sep=},
    ] 
    \addlegendentry{$\Delta u = \num{0,1}$},
    \addlegendentry{$\Delta u = \num{0,2}$}
    \addlegendentry{$\Delta u = \num{0,5}$},
    	\addlegendimage{no markers,green}
	\addlegendimage{no markers,red}
	\addlegendimage{no markers,yellow}
    \addplot[red, semithick, thick] file{Rysunki/PROJ1_2Uskok=2.1.txt};
    \addplot[green, semithick, thick] file{Rysunki/PROJ1_2Uskok=2.2.txt};
    \addplot[yellow, semithick, thick] file{Rysunki/PROJ1_2Uskok=2.5.txt};


    
    \end{axis}
    \end{tikzpicture}
    \caption{Odpowiedzi procesu na skokowe zmiany sterowanie}
    \label{zad2_porow_odp_skok}
\end{figure}

W celu wyznaczenia odpowiedzi skokowej obiekt, znajdujący się w punkcie pracy (tzn. $U_{\mathrm{pp}}= \num{2.0}$,$Y_{\mathrm{pp}}= \num{0.8}$) pobudzoną różną zmianą wartoci sterowań. Wykres \ref{zad2_skoki_wykres} przedstawia odpowiedź obiektu na jego różne wartosci.

\section{Wyznaczanie charakterystyki statycznej procesu}
\label{zad2_char_stat}
Aby wyznaczyć charakterystykę statyczną procesu przeprowadzono analogiczne działania co w rozdziale \ref{zad1}. Tym razem przy użyciu skryptu \verb+PROJ1_2.m +dla wielu wartosci $U_{\mathrm{pp}}$ wyznaczono odpowiadające im $Y_{\mathrm{pp}}$ oraz z ich pomocą utworzono wykres \ref{zad2_char_stat}. Jak widać charakterystyka statyczna obiektu jest liniowa, a co za tym idzie obiekt jest liniowy.

\begin{figure}[b]   
     \label{zad2_char_stat}
    \centering
    \begin{tikzpicture}
    \begin{axis}[
    width=0.98\textwidth,
    ymin=0.3,ymax=1.3,xmin=1.2,xmax=2.8,
    xlabel={$u$},
    ylabel={$y(u)$},
    ytick={0.3, 0.5, 0.7, 0.9, 1.1,1.3},
    xtick={1.2,1.6,2.0,2.4,2.8},
    legend pos=south east,
    y tick label style={/pgf/number format/1000 sep=},
    ]
    \addplot[red, semithick] file{rysunki/PROJ1_2char_stat.txt};
    \legend{$y(u)$}
    \end{axis}
    \end{tikzpicture}
    \caption{Charakterystka statyczna $y(u)$ symulowanego procesu}

\end{figure}

\section{Wzmocnienie statyczne}
\label{zad2_wzmocnienie}
Wzmocnienie statyczne, czyli stosunek pomiędzy zmianą wartosci wyjscia i zmianą wartosci sterowania w stanie ustalonym. Aby ją wyznaczyć można na przykład znaleźć nachylenie charakterystyki statycznej do osi OX, czyli np.:

\begin{equation}
K_{\mathrm{stat}} = \frac{y(U_{\mathrm{max}})- y(U_{\mathrm{min}})}{U_{\mathrm{max}}- U_{\mathrm{min}}}
\label{zad2_wzm_statyczne_wzor}
\end{equation}

W przypadku tak wykreślonej charakterystyki, wzmocnienie statyczne jest równe tangensowi kąta $\alpha$
pomiędzy prostą a osią $OX$. 
\begin{equation}
K_{\mathrm{stat}} = \frac{1,239- 0,361}{2,8- 1,2}\approx 0,549
\label{zad2_wzm_statyczne_podstawienie}
\end{equation}
